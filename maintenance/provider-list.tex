\documentclass[../main/main]{subfiles}
\begin{document}
\newpage
\section{Provider List}
\label{sec:provider-list}

Upon clicking on the \textbf{Provider List} button,
the area on the right-hand side of the menu will be replaced with the
\emph{Provider List} screen.

\includegraphics[width=\textwidth,keepaspectratio,center]{%
  ../maintenance/imgs/provider-list%
}

From the \emph{Provider List} screen, you can view and modify existing provider
lists as well as add new provider lists.

A provider list is simply a group of providers.

All existing provider lists will show up in the \emph{Provider Lists} section
at the top of the screen.

If you click on a provider list in the table, the corresponding providers in
the list will be shown in the \emph{Selected Provider List} section below.

You can also add a new provider list by clicking on the \textbf{Add} button at
the top right of the screen.

Fields marked in yellow colour are mandatory to be filled in before saving.

To add one or more providers to the list, you can click on the \textbf{>>}
button which will add all the available providers or the \textbf{>} button
which will add the currently selected provider(s) in the left table (Available
Provider). The providers will be removed from the table on the left and added to
the table on the right.

\pagebreak
To remove one or more providers from the list, you can click on the
\textbf{<<} button which will remove all the selected providers or the
\textbf{<} button which will remove the currently selected provider(s) in the
right table (Selected Provider). The providers will be removed from the table
on the right and added back to the table on the left.

When you are done modifying the provider list, click on the \textbf{Save}
button to save the changes.

The \emph{Provider List} screen can be closed by clicking on the
\textbf{Close} button at the bottom right of the screen. If there are any
unsaved modifications, a confirmation box will popup to confirm that you want
to close without saving. Click on the \textbf{Yes} button to close the screen
or on the \textbf{No} button to stay in the screen.

\end{document}
