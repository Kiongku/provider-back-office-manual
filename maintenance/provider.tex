\documentclass[../main/main]{subfiles}
\begin{document}
\newpage
\section{Provider}
\label{sec:provider}

Upon clicking on the \textbf{Provider} button,
the area on the right-hand side of the menu will be replaced with the
\emph{Provider} screen.

\includegraphics[width=\textwidth,keepaspectratio,center]{%
  ../maintenance/imgs/provider%
}

From the \emph{Provider} screen, you can view, modify or delete existing
providers as well as add new providers.

In the \emph{Provider Search} section at the top of the \emph{Provider} screen,
you can specify search filtering by provider code or/and name before
clicking on the \textbf{searcH} button to search for the matching providers.
If the fields are left blank, the search will return all providers.

\includegraphics[width=0.8\textwidth,keepaspectratio,center]{%
  ../maintenance/imgs/provider-search%
}

After the search is complete, providers found will be displayed in the
\emph{Provider List} section.

\includegraphics[width=0.8\textwidth,keepaspectratio,center]{%
  ../maintenance/imgs/provider-search-list%
}

If there is no provider matching the search criteria, a `No Record Found!'
message will appear as a popup window.

\pagebreak
When a provider is selected in the \emph{Provider List} section, the details
will appear in the \emph{Provider} section below.

\includegraphics[width=0.8\textwidth,keepaspectratio,center]{%
  ../maintenance/imgs/provider-details%
}

You can also add a new provider by clicking on the \textbf{Add} button at the
top right of the screen.

Fields marked in yellow colour are mandatory to be filled in before saving.

The provider settings for the Booking Module can be modified by clicking on the
\textbf{Booking Settings} button. The \emph{Booking Settings} window will
appear as a popup window.

\includegraphics[width=0.5\textwidth,keepaspectratio,center]{%
  ../maintenance/imgs/provider-booking-settings%
}

There are certain restrictions on the `Slot' and `Time Interval' fields.
The `Slot' field only accepts input from 1 to 3 inclusive while the `Time
Interval' field only accepts 5, 10, 15, 20, 30, and 60 as input.

Click on the \textbf{Apply} button when you are done or on the \textbf{Close}
button if you do not wish to modify the booking settings.

You can set the specialty of the provider in the \emph{Specialty} section.
Click on the \textbf{Add} button to add a new specialty. A new row will appear
in the table with a dropbox to select available specialties. If you wish to
remove the specialty instead, click on the \textbf{Delete} button on the right
of the specialty.

\pagebreak
You can set the attending clinic of the provider in the \emph{Attending Clinic}
section. Click on the \textbf{Add} button to add a new attending clinic. A new
row will apear in the table with a dropbox to select available clinics. If you
wish to remove the attending clinic instead, click on the \textbf{Delete} button
on the right of the attending clinic.

You can set the attending hospital of the provider in the \emph{Attending
Hospital} section. Click on the \textbf{Add} button to add a new attending
hospital. A new row will appear in the table with a dropbox to select available
hospitals. You can also fill in the Provider Code assigned to the provider in
the hospital. If you wish to remove the attending hospital instead, click on
the \textbf{Delete} button on the right of the attending hospital.

When you are done modifying the provider details, click on the \textbf{Save}
button to save the changes.

If you want to delete a provider, click on the \textbf{Delete} button at the
bottom left-hand side of the screen. A confirmation box will popup.
Click on the \textbf{Yes} button if you want to confirm the removal of the
provider or on the \textbf{No} button if you do not wish to delete it anymore.

The \emph{Provider} screen can be closed by clicking on the
\textbf{Close} button at the bottom right of the screen. If there are any
unsaved modifications, a confirmation box will popup to confirm that you want
to close without saving. Click on the \textbf{Yes} button to close the screen
or on the \textbf{No} button to stay in the screen.

\end{document}
