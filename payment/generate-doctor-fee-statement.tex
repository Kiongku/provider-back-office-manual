\documentclass[../main/main]{subfiles}
\begin{document}
\newpage
\section{Generate Doctor Fee Statement}
\label{sec:generate-doctor-fee-statement}

Upon clicking on the \textbf{Generate Doctor Fee Statement} button,
the area on the right-hand side of the menu will be replaced with the
\emph{Generate Doctor Fee Statement} screen.

\includegraphics[width=\textwidth,keepaspectratio,center]{%
  ../payment/imgs/generate-doctor-fee-statement%
}

From the \emph{Generate Doctor Fee Statement} screen, you can search for
providers with pending payments and select them for generating statements.

Click on the \textbf{Search} button to search for providers with pending
payments. The providers will be shown in the \emph{Provider List} section.
If no matching providers were found, a `No record found!' message will appear.

\includegraphics[width=0.8\textwidth,keepaspectratio,center]{%
  ../payment/imgs/generate-provider-list%
}

To select or deselect all providers, the checkbox named \emph{Select/ deselect
all} found at the bottom left of the screen can be used for this purpose.

To select or deselect individual providers, the checkbox on the left of each
provider in the \emph{Provider List} can be used.

You can also specify the format of the statement as `Simple' or `Detailed'
format from the dropdown box in the `Statement Format' column to the right
of the provider.

When you have selected the providers that you want to generate statements for,
click on the \textbf{Generate} button at the bottom right of the screen.

Each statement will appear as a popup window containing the pdf version of
the statement. The statement can be printed from that window.

The \emph{Generate Doctor Fee Statement} screen can be closed by clicking on the
\textbf{Close} button at the bottom right of the screen.

\end{document}
