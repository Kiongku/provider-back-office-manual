\documentclass[../main/main]{subfiles}
\begin{document}
\newpage
\section{Settlement}
\label{sec:settlement}

Upon clicking on the \textbf{Settlement} button,
the area on the right-hand side of the menu will be replaced with the
\emph{Settlement} screen.

\includegraphics[width=\textwidth,keepaspectratio,center]{%
  ../billing-settlement/imgs/settlement%
}

In the \emph{Settlement} screen, invoices can be marked as settled. Invoices
outstanding amount can also be adjusted in this screen.

The top section \emph{Criteria} is used to refine the search for invoices
that will be displayed in the \emph{Invoices} section.
The search can be filtered by Company, Invoice number or Invoice period.
You also have the option to display settled transactions from the search.
By default, only unsettled transactions are displayed.
Click on the \textbf{Search} button to retrieve the invoices matching the search
criteria. If you do not input any search option, all invoices will be retrieved.

\includegraphics[width=0.8\textwidth,keepaspectratio,center]{%
  ../billing-settlement/imgs/settlement-invoices%
}

\pagebreak
There are four columns that are of importance in the list, namely:-
\begin{description}
\item[Total Amount] \hfill \\
The total amount due for the invoice.
\item[Settled Amount] \hfill \\
The amount that has been paid already previously.
\item[Payment Amount] \hfill \\
The amount that is currently being paid.
\item[Outstanding Amount] \hfill \\
The amount that has not been settled yet.
\end{description}

To mark the whole invoice as paid, the checkbox in the \emph{Fully Paid} column
on the left of the invoice in \emph{Invoices} section can be checked.

To access the invoice details, the emph{\ldots} button at the right of invoice
can be clicked on. The \emph{Settlement} screen will be replaced by the
\emph{Invoice Detail} screen. This screen contains the treatments included
in the invoice.

\includegraphics[width=0.8\textwidth,keepaspectratio,center]{%
  ../billing-settlement/imgs/invoice-details%
}

From the \emph{Invoice Detail}, you can select to pay the invoice
per treatment by ticking the checkbox in the \emph{Fully Paid} column on
the left of the treatment being paid.

The treatment amount can also be adjusted by clicking on the
\textbf{Adjustment} button on the right of the treatment. The
\emph{Invoice Adjustment} will appear as a popup window.

\includegraphics[width=0.8\textwidth,keepaspectratio,center]{%
  ../billing-settlement/imgs/invoice-adjustment%
}

The new amount can be set in the `New Total Amount' field. The difference
between the original amount and new amount will be shown in the `Amount
Adjusted' box. If the amount paid to provider needs to also be adjusted,
you can tick the checkbox `Adjust Payment to Provider'. A reason needs also
to be supplied into the `Adjustment Reason' text box.

The \emph{Adjustment History} section contains a list of previous adjustment
done.

After finishing the required adjustment, click on the \textbf{Save Adjustment}
button to save the changes. The new adjustment will be shown in the
\emph{Adjustment History} section.

To close the \emph{Invoice Adjustment} window and go back to the
\emph{Invoice Detail} screen, click on the \textbf{Close} button at the bottom
right of the window.

You can click on the \emph{Back} button at the bottom right of the screen to
return to the \emph{Settlement} screen.

When you are done with the \emph{Settlement} screen, you can save your changes
by clicking on the \textbf{Save} button at the bottom right of the screen.

The \emph{Settlement} screen can be closed by clicking on the
\textbf{Close} button at the bottom right of the screen.

\end{document}
